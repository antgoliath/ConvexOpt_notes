\documentclass[answers]{exam}

% ------------------------------------------------------------------------------ %
% -----------------------      Base for every .tex file   ---------------------- %
% ------------------------------------------------------------------------------ %

\usepackage[dvipsnames]{xcolor}
\usepackage{mathtools}
\usepackage{amssymb}
\usepackage{amsthm}
\usepackage{amsmath}
\usepackage{framed}
\usepackage{wasysym}
\usepackage{geometry}
\usepackage{cancel}
\usepackage{blindtext}
\usepackage{pgfplots}
\usepackage{graphicx}
\usepackage{lastpage}
\usepackage[most]{tcolorbox} 
\usepackage{multicol}
\usepackage{soul}
\usepackage{listings}
\usepackage{algorithm}
\usepackage{algorithmic}
\usepackage{booktabs}
\usepackage{tikz}
\usepackage{pifont}

% Libraries
\usetikzlibrary{shapes,shapes.geometric, positioning, arrows}

\geometry{%
	left=15mm,
	right=15mm,
	top=25mm,
	bottom=25mm,
	bindingoffset=0mm,
	headheight=30pt,% output from geometry tells you what this needs to be set to as a minimum
}

% Header and Footer
\pagestyle{headandfoot}
\firstpageheadrule
\runningheadrule
\firstpageheader{Convex Optimization}{\today}{Jonathan Schnell}
\runningheader{Convex Optimization}{}{Jonathan Schnell}
\firstpagefooter{}{Page \thepage\ of \numpages}{}
\runningfooter{}{Page \thepage\ of \numpages}{}

% Commands
\newcommand{\imp}[1]{\ul{\textbf{#1}}}
\newcommand{\dproduct}[1]{\left\langle #1 \right\rangle}
\newcommand{\norm}[1]{\left\lVert #1 \right\rVert}
\renewcommand{\vector}[1]{\begin{pmatrix} #1 \end{pmatrix}}
\newcommand{\abs}[1]{\left| #1 \right|}
\newcommand{\floor}[1]{\lfloor #1 \rfloor}
\newcommand{\ceil}[1]{\lceil #1 \rceil}
\newcommand{\fracpart}[2]{\frac{\partial #1}{\partial #2}}
\newcommand{\set}[2]{\left\{#1 \ \middle|\ #2\right\}}
\renewcommand{\hat}[1]{\widehat{#1}}

\newcommand{\Ker}{\operatorname{Ker}}
\renewcommand{\Im}{\operatorname{Im}}
\renewcommand{\Re}{\operatorname{Re}}
\renewcommand{\dim}{\operatorname{dim}}
\renewcommand{\div}{\operatorname{div}}
\newcommand{\rot}{\operatorname{rot}}
\newcommand{\grad}{\operatorname{grad}}
\newcommand{\vol}{\operatorname{vol}}
\newcommand{\supp}{\operatorname{supp}}
\renewcommand{\div}{\operatorname{div}}
\newcommand*{\vertbar}{\rule[-1ex]{0.5pt}{2.5ex}}
\newcommand*{\horzbar}{\rule[.5ex]{2.5ex}{0.5pt}}

\theoremstyle{definition}
\newtheorem*{definition}{Definition}
\newtheorem*{beispiel}{Beispiel}
\newtheorem*{remark}{Remark}

\theoremstyle{plain}
\newtheorem*{proposition}{Proposition}
\newtheorem*{satz}{Satz}
\newtheorem*{korollar}{Korollar}
\newtheorem*{lemma}{Lemma}
\newtheorem*{theorem}{Theorem}


% Quote
\newtcolorbox{zitat}[1]{%
	colback=lightGray,
	grow to right by=-10mm,
	grow to left by=-10mm, 
	boxrule=0pt,
	boxsep=0pt,
	breakable,
	enhanced jigsaw,
	borderline west={4pt}{0pt}{gray},
	#1
}

% Use colors in equations
\newcommand{\highlight}[2]{\colorbox{#1}{$#2$}}%
\definecolor{lightGray}{gray}{0.9} 

% To add shortcut of script Letters in Equations
\newcommand{\s}[1]{\mathcal{#1}}
\newcommand*\circled[1]{\tikz[baseline=(char.base)]{
            \node[shape=circle,draw,inner sep=2pt] (char) {#1};}}
\newcommand{\cmark}{\ding{51}}
\newcommand{\xmark}{\ding{55}}


\newenvironment{claim}[1]{
		\par\noindent
		\textbf{Claim.} #1
		\begin{tcolorbox}[blanker, top=3mm, bottom=3mm, left=3mm, borderline west={1pt}{0mm}{black}]
		\noindent\textit{Proof of Claim.} 
}{
	\hfill$\blacksquare$	
	\end{tcolorbox}\noindent
}

% To add shortcut of number's set Z
\newcommand*{\Z}{\mathbb{Z}}
\newcommand*{\N}{\mathbb{N}}
\newcommand*{\R}{\mathbb{R}}
\newcommand*{\Q}{\mathbb{Q}}
\newcommand*{\C}{\mathbb{C}}
\newcommand*{\F}{\mathbb{F}}
\newcommand*{\K}{\mathbb{K}}

% To add shortcut of empty set
\renewcommand*{\o}{\varnothing}
\pgfplotsset{compat=1.9}

\everymath{\displaystyle}

% Line-Height
\linespread{1.15}

\graphicspath{{Files/}}

% ------------------------------

\begin{document}

	$ $
	\begin{center}
		\huge \textbf{Exercise session notes - Week 11}  \\ \vspace*{3mm}
        \Large{Semidefinite Programs + Applications}
	\end{center}
	$ $\\

    \section{Application: Markowitz portfolio optimization}

    In this problem we have a budget $B$ that we want to invest in assets $1,\ldots, n$. At time $t = 0$ we know the prices of the assets as
    $$ p_1^0,\ldots , p_n^0 $$
    After some period of time we may know some predictions of the prices, and at time $t = T$ we know the distribution of the prices as 
    $$ P_1^T,\ldots , P_n^T $$
    which are random variables. In order to model the problem as a mathematical problem we define the Rate of Return:
    $$ R_i = \frac{P_i^T}{p_i^0} - 1 $$
    which essentially gives us the amount that we receive at $T$ by buying asset $i$. (if $R_i = 0$ then we receive the same amount, if $R_i = 1$ then we double the amount, if $R_i = -\tfrac{1}{2}$ then we lose half of the amount). \\ 
    We then define our decision variables $x_i = $ fraction of the budget $B$ spent on asset $i$ (the vector $x$ is called portfolio). Then the final revenue/loss is the random variable 
    $$ R^\top x = \sum_{i} R_i \cdot x_i $$
    This RV has expected value 
    $$ \mu^\top x := \sum_{i} x_i\cdot  \mathbb{E}[R_i]  $$
    and variance 
    $$ x^\top \Sigma x := \sum_{i,j} \operatorname*{Cov}(R_i, R_j)x_ix_j $$
    Finally we can write our mathematical program as follows: 
    \begin{itemize}
        \item \textbf{Goal:} Minimize risk with lower bound on profit, as QP.
        \begin{align*}
            \min\quad x^\top \Sigma x& \\ 
            \text{s.t}\quad \mu^\top x&\geq b \\ 
            \textbf{1}^\top x &= 1 
        \end{align*}
        \item \textbf{Goal:} Maximize profit with upper bound on risk, as QCQP.
        \begin{align*}
            \max\quad \mu^\top x& \\ 
            \text{s.t}\quad x^\top \Sigma x&\leq \gamma^2 \\ 
            \textbf{1}^\top x &= 1 
        \end{align*}
        \item \textbf{Goal:} Maximize utility function (balance between profit and risk), as SOCP.
        \begin{align*}
            \max\quad \mu^\top x - \delta \sqrt{x^\top \Sigma x}& \\ 
            \text{s.t}\quad \textbf{1}^\top x &= 1 
        \end{align*}
    \end{itemize}

\section{Application: Max-Cut Problem}
    We firstly recall the definition of a SemiDefinite Program:
    \begin{align*}
        \min\quad \operatorname{Tr}(CX)& \\ 
        \text{s.t.}\quad \operatorname{Tr}(A_iX)& = b \\ 
        X&\succeq 0
    \end{align*}
    As the objective function and the constraints are affine, and $X\succeq 0$ is equivalent to $X$ being in the positive semidefinite cone, we get SDP $\subseteq$ CP. In the lecture notes it is also proven SOCP $\subseteq$ CP, and thus we concluded the inclusions related to convex programs.\\

    In the Max-Cut Problem we are given a graph $G=(V,E)$, with edge weights $w_e$ for $e\in E$ and $\sum_{e} w_e = 1$. The goal is to find a subset of vertices $S\subseteq V$ with cut value 
    $$ c(S) := \sum_{e\in \delta(S)} w_e $$
    maximized. For example the cut in the following graph has cut value $4$ and it is the maximum that we can achieve
    \begin{center}
        \includegraphics*[width=0.3\textwidth]{MaxCut.PNG}
    \end{center}
    Note that for general graphs this problem is NP-hard, on the other hand for bipartite graphs and complete graphs this problem becomes trivial. There is also a theorem that gives us the existence of a cut with cut value $ \geq \tfrac{1}{2}$ (see Ch. 7.5 in Lecture notes). 

\subsection{Semidefinite Formulation}
    We want to solve the mathematical problem 
    $$ \max \quad \sum_{e\in \delta(S)} w_e\quad \text{s.t.}\quad S\subseteq V $$
    To do that we first define the decision variables $x\in \{-1,1\}^{V}$ with 
    $$ x_v = 1 \iff v\in S $$
    Therefore we can write an equivalent program as 
    \begin{align*}
        \max\quad  \sum_{uv\in E} w_{uv} \cdot \left(\frac{1-x_ux_v}{2}\right) \quad \text{s.t.}\quad x\in \{-1,1\}^V
    \end{align*}
    Let us define the matrix $X = (x_u\cdot x_v)_{u,v\in V} \in \{-1,1\}^n\times n$, then we can write the program as 
    \begin{align*}
        \max \quad \operatorname{Tr}(WX) &= \sum_{uv} X_{u,v} W_{u,v} \\ 
        \text{s.t.}\quad X_{ii} &= 1 \\ 
        X &= xx^\top \\ 
        x&\in \{-1,1\}^n 
    \end{align*}
    Finally we can drop the last two constraints and we get the SDP relaxation 
    \begin{align*}
        \max \quad \operatorname{Tr}(WX) &= \sum_{uv} X_{u,v} W_{u,v} \\ 
        \text{s.t.}\quad X_{ii} &= 1 \\ 
        X&\succeq 0 
    \end{align*}
    By solving this problem we end up with an approximation of $0.87856\ldots$ of the value of the max-cut. 





\end{document}