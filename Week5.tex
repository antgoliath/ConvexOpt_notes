\documentclass[answers]{exam}

\input{header.tex}

% ------------------------------

\begin{document}

	$ $
	\begin{center}
		\huge \textbf{Exercise session notes - Week 5}  \\ \vspace*{3mm}
        \Large{Strong Duality + KKT points}
	\end{center}
	$ $\\

    \noindent Let's consider a general mathematical program
    \begin{align*}
        \min \quad f(x) & \\ 
        g_i(x) &\leq 0 \quad\forall i\\
        h_j(x) &= 0 \quad\forall j
    \end{align*}
    Last week we constructed the Lagrange Dual Program as follows 
    \begin{align*}
        \max \quad \hat{L}(\lambda, \nu)& \\ 
        \lambda \geq 0
    \end{align*}
    where the objective function is the Lagrange dual function, defined as 
    $$ \hat{L}(\lambda, \nu) = \inf_x\Big\{ f(x) + \sum \lambda_i g_i(x) + \sum \nu_j h_j(x)\Big\} $$
    We now list some properties about duality. \\
    \paragraph{Proposition.} (Weak Duality) For any program we have 
    $$\max \hat{L}(\lambda, \nu) \leq \min f(x)$$
    \paragraph{Definition.} (Strong Duality) We say strong duality holds if the duality gap is $0$, i.e.  
    $$\max \hat{L}(\lambda, \nu) = \min f(x)$$
    \paragraph{Proposition.} (Complementary Slackness) If strong duality holds, then for any optimal primal-dual solution $(x^*, \lambda^*, \nu^*)$ we have 
    $$\lambda_i^* \cdot g_i(x^*) = 0\quad\quad \forall i $$
    This means that for each $i$, either the $i$th primal constraint is tight ($g_i(x) = 0$) or the $i$th dual constraint is tight ($\lambda_i = 0$), or both of them are tight. 
    \paragraph{Definition.} (KKT-conditions) A point $(x,\lambda, \nu)$ is a KKT-point if:
    \begin{alignat*}{3}
        \circled{1}\quad &\text{Primal feasible}\quad &&:\quad g_i(x)\leq 0;\ h_j(x) = 0 \\ 
        \circled{2}\quad &\text{Dual feasible}\quad &&: \quad \lambda_i\geq 0 \\
        \circled{3}\quad &\text{Comp. slackness}\quad &&: \quad \lambda_i\cdot g_i(x) = 0 \\ 
        \circled{4}\quad &\text{Gradient vanishes}\quad &&:\quad \nabla_x L(x,\lambda, \nu) = 0
    \end{alignat*}

    \paragraph{Proposition.} For \underline{any} mathematical program: 
    $$ \text{Strong Duality holds} + \text{$(x,\lambda,\nu)$ is optimal} \implies \text{$(x, \lambda, \nu)$ is KKT point} $$
    \paragraph{Proposition.} For \underline{convex} mathematical programs: 
    $$ \text{Strong Duality holds} + \text{$(x,\lambda,\nu)$ is optimal} \iff \text{$(x, \lambda, \nu)$ is KKT point} $$

    Now we show an example of a convex mathematical program for which Strong duality does not hold. 

    \paragraph{Example.} Consider the following program (P)
    \begin{align*}
        \min \quad e^{-x} & \\ 
        \frac{x^2}{y} &\leq 0
    \end{align*}
    where we set the domain as $\s{D} = \set{(x,y)}{y > 0}$. 
    % The feasibility set is the following
    % \begin{center}
    %     \includegraphics[width=0.25\textwidth]{FeasibleSet.png}
    % \end{center}
    Note that in this domain, the program is a convex program (Exercise). Moreover the optimal value is $\highlight{lightGray}{f^* = 1}$. We now compute the dual problem: Let $\lambda \geq 0$, then
    \begin{align*}
        &L(x,y,\lambda) = e^{-x} + \lambda \frac{x^2}{y} \\
        \implies & \hat{L}(\lambda) = \inf_{x,y} \Big\{e^{-x} + \lambda \frac{x^2}{y}\Big\}
    \end{align*}
    Since $y$ is positive, we have $L(x,y,\lambda) \geq 0$. Moreover if we let $x = n$, $y = n^3$ for $n\in \N$ and tend $n\to \infty$ we obtain 
    $$ L(n, n^3, \lambda) = e^{-n} + \lambda\frac{1}{n} \to 0 $$
    In particular we have $\hat{L}(\lambda) = 0$. With this, we can write the dual program (D) of (P) by 
    \begin{align*}
        \max \quad \hat{L}(\lambda) &= 0 \\ 
        \lambda &\geq 0
    \end{align*}
    with optimal value $\highlight{lightGray}{\hat{L}^* = 0}$. For this pair of programs we have a duality gap of $1$, so strong duality does not hold. By the previous proposition there exists no KKT point, so let's see what goes wrong. Consider a point $(x, y, \lambda)$ in the domain of the primal/dual programs and assume it is a KKT-point.
    \begin{itemize}
        \item[\circled{1}] We have $\tfrac{x^2}{y} = 0$ so $x = 0$. $\qquad \text{\cmark}$
        \item[\circled{2}] We have $\lambda \geq 0$. $\qquad \text{\cmark}$
        \item[\circled{3}] We have $\lambda\tfrac{x^2}{y} = \lambda\tfrac{0}{y} = 0$. $\qquad \text{\cmark}$
        \item[\circled{4}] We have $\nabla_{x,y} L = 0$ so
        $$ \frac{\partial}{\partial y} L = 0 - \lambda \frac{x^2}{y^2} = 0 \quad\text{\cmark}  $$
        and 
        $$ \frac{\partial}{\partial x} L = -e^{-x} + 2\lambda \frac{x}{y} = -e^{0} = -1 \neq 0\quad \text{\xmark}  $$
    \end{itemize}
    Therefore there are no KKT-points for this problem.

\end{document}
